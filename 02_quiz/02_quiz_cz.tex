\documentclass[9pt]{beamer}
\usepackage[utf8]{inputenc}
%\usepackage[czech]{babel}
%\usepackage{slashed}
%\usepackage{axodraw}
%\usepackage{xcolor}
\usepackage{hyperref}
\hypersetup{unicode=true, colorlinks=true}
\usepackage{graphicx}
\usepackage{xspace}
%\usepackage{caption}
%\usepackage{subcaption}
%\usetheme{Warsaw}
\usepackage{variables}

\usepackage{listings}
\usepackage{xcolor}

\lstset{
    language=Python,
    basicstyle=\ttfamily\small,
    columns=flexible,         % Improves character spacing
    showspaces=false,          % Removes the u-shaped space symbols
    showstringspaces=false,    % Removes symbols inside 'strings'
    showtabs=false,            % Removes symbols for tabs
    breaklines=true,           % Wraps long lines
    keywordstyle=\color{blue},
    commentstyle=\color{gray},
    stringstyle=\color{orange}
}

\newenvironment{narrow}[2]{%
  \begin{list}{}{%
    \setlength{\topsep}{0pt}%
    \setlength{\leftmargin}{#1}%
    \setlength{\rightmargin}{#2}%
    \setlength{\listparindent}{\parindent}%
    \setlength{\itemindent}{\parindent}%
    \setlength{\parsep}{\parskip}%
  }%
  \item[]
}{\end{list}}

\title[]{Kvíz}
\author[Vojtech Pleskot]{Vojtech Pleskot\inst{1}} %use \and to separate two authors
\institute[]{UK Praha\inst{1}}
\date{}

%\setbeamertemplate{headline}
%{
%  \begin{beamercolorbox}[wd=\paperwidth,ht=3.5ex,leftskip=.3cm,rightskip=.3cm]{section in head/foot}
%    \vskip2pt\insertsection \hskip20pt \insertsubsection\vskip2pt
%  \end{beamercolorbox}
%}

\setbeamertemplate{footline}
{
%  \hbox{\begin{beamercolorbox}[wd=.5\paperwidth,ht=2.5ex,dp=1.125ex,leftskip=.3cm plus1fill,rightskip=.3cm]{author in head/foot}
%    \usebeamerfont{author in head/foot}\insertauthor
%  \end{beamercolorbox}
  \begin{beamercolorbox}[wd=\paperwidth,ht=2.0ex,dp=3.0ex,leftskip=.3cm,rightskip=.3cm plus1fil]{title in head/foot}
    \usebeamerfont{title in head/foot} \hfill \insertframenumber/\inserttotalframenumber
  \end{beamercolorbox}%}
  \vskip0pt
}

\begin{document}
\begin{frame}
  \titlepage
\end{frame}

% Toto je jednoduchý kvíz s jednou otázkou na snímek. Správná odpověď je odhalena na dalším snímku.
% Pro každou otázku jsou čtyři až šest možných odpovědí.
% Otázky jsou očíslovány od A do F.
% Může existovat více správných odpovědí na každou otázku, ale alespoň jedna z odpovědí je správná.
% Test zkouší znalosti univerzitních studentů programovacího jazyka Python a jeho knihoven, stejně jako jejich schopnost jej použít pro analýzu dat a vizualizaci.
% Konkrétně je testována práce s řetězci, seznamy, slovníky, množinami, n-ticemi, funkcemi, třídami, moduly, balíčky a knihovnami jako NumPy, Matplotlib, SciPy a uncertainties.


% Otázka na definici funkce
\begin{frame}[fragile]
  Jaká je hodnota proměnné \texttt{a} po provedení následujícího kódu?
\begin{lstlisting}
def func(x):
    return x * 2
a = func(3)
\end{lstlisting}
\begin{itemize}
\item A) 5
\item B) 6
\item C) 9
\item D) 1.5
\end{itemize}
\end{frame}

% Otázka na indexování seznamu
\begin{frame}[fragile]
  Jaký je výstup následujícího kódu?
\begin{lstlisting}
my_list = [1, 2, 3]
print(my_list[2])
\end{lstlisting}
\begin{itemize}
\item A) [3]
\item B) [1, 2]
\item C) 3
\item D) IndexError: list index out of range
\end{itemize}
\end{frame}

% Otázka na formátování řetězců
\begin{frame}[fragile]
  Mějme proměnnou \texttt{x}, jejíž hodnota je 10. Který řetězec je roven "The value of x is 10"?
\begin{itemize}
\item A) "The value of x is " + str(x)
\item B) "The value of x is 10"
\item C) "The value of x is " + x
\item D) f"The value of x is {x}"
\item E) "The value of x is {}".format(x)
\end{itemize}
\end{frame}

% Otázka na slovník
\begin{frame}[fragile]
  Jaký je výstup následujícího kódu?
\begin{lstlisting}
my_dict = {'a': 1, 'b': 2, 'c': 3}
print(my_dict['b'])
\end{lstlisting}
\begin{itemize}
\item A) 'b'
\item B) 2
\item C) {'b': 2}
\item D) KeyError: 'b'
\end{itemize}
\end{frame}

% Otázka na rozsah proměnné
\begin{frame}[fragile]
  Jaká je hodnota \texttt{x} po provedení následujícího kódu?
\begin{lstlisting}
x = 10
def func():
    x = 5
    return x
a = func()
\end{lstlisting}
\begin{itemize}
\item A) None
\item B) 5
\item C) nan
\item D) 10
\end{itemize}
\end{frame}

% Otázka na operace s množinami
\begin{frame}[fragile]
  Jaký je výstup následujícího kódu?
\begin{lstlisting}
my_set = {1, 2, 3}
my_set.add(3)
print(my_set)
\end{lstlisting}
\begin{itemize}
\item A) {1, 2, 3, 3}
\item B) AttributeError: 'set' object has no attribute 'add'
\item C) {1, 2, 3}
\item D) 'Jak vám dupou králíci?'
\end{itemize}
\end{frame}

% Otázka na řezy seznamu
\begin{frame}[fragile]
  Jaký je výstup následujícího kódu?
\begin{lstlisting}
my_list = [1, 2, 3, 4, 5]
print(my_list[1:4])
\end{lstlisting}
\begin{itemize}
\item A) [1, 2, 3]
\item B) [2, 3, 4]
\item C) [3, 4, 5]
\item D) [1, 2, 3, 4]
\item E) [2, 3, 4, 5]
\end{itemize}
\end{frame}

% Otázka na slovníkové generátory
\begin{frame}[fragile]
  Jaký je výstup následujícího kódu?
\begin{lstlisting}my_dict = {x: x**2 for x in range(5)}
print(my_dict)
\end{lstlisting}
\begin{itemize}
\item A) {1: 1, 2: 4, 3: 9, 4: 16}
\item B) {1: 1, 2: 4, 3: 9, 4: 16, 5: 25}
\item C) {0: 0, 1: 1, 2: 4, 3: 9, 4: 16}
\item D) {0: 0, 1: 1, 2: 4, 3: 9, 4: 16, 5: 25}
\item E) {(0, 1, 2, 3, 4): (0, 1, 4, 9, 16)}
\end{itemize}
\end{frame}

% Otázka na rozbalení n-tice
\begin{frame}[fragile]
  Jaký je výstup následujícího kódu?
\begin{lstlisting}
def func():
    return 1, 2
a, b = func()
print(a, b)
\end{lstlisting}
\begin{itemize}
\item A) (1, 2)
\item B) 1 2
\item C) 1, 2
\item D) ValueError: too many values to unpack
\end{itemize}
\end{frame}

% Otázka na čtení textového souboru
\begin{frame}[fragile]
  Jak získat obsah souboru 'file.txt' načtený do proměnné \texttt{data}?
\begin{itemize}
\item A) print("Sezame, otevři se!")
\item B) data = open('file.txt', 'r')
\item C) with open('file.txt', 'r') as f: data = f.read()
\item D) f = 'file.txt'; data = f.read()
\item E) f = 'file.txt'; for line in f: data += line
\end{itemize}
\end{frame}

% Otázka na import modulu
\begin{frame}[fragile]
  Jak vypočítat \texttt{sin(1)} pomocí modulu \texttt{math}?
\begin{itemize}
\item A) import math; sin(1)
\item B) from math import sin; sin(1)
\item C) import math as m; math.sin(1)
\item D) import math; math.sin(1)
\item E) chodit do kurzu NOFY084, bedlivě poslouchat a udělat všechny úkoly
\end{itemize}
\end{frame}

% Otázka na sčítání NumPy polí
\begin{frame}[fragile]
  Jaký je výstup následujícího kódu?
\begin{lstlisting}
import numpy as np
a = np.array([1, 2, 3])
b = np.array([4, 5, 6])
print(a + b)
\end{lstlisting}
\begin{itemize}
  \item A) [1 2 3] + [4 5 6]
  \item B) [1 2 3 4 5 6]
  \item C) [4 5 6 1 2 3]
  \item D) [5 7 9]
\end{itemize}
\end{frame}

% Otázka na násobení NumPy polí
\begin{frame}[fragile]
  Jaký je výstup následujícího kódu?
\begin{lstlisting}
  import numpy as np
  a = np.array([1, 2, 3])
  b = np.array([1, 2, 3])
  print(a * b)
\end{lstlisting}
\begin{itemize}
\item A)
\begin{lstlisting}
[[1 2 3]
 [2 4 6]
 [3 6 9]]
\end{lstlisting}
\item B) 14
\item C) [1 4 9]
\item D) [1 2 3] * [4 5 6]
\end{itemize}
\end{frame}

% Otázka na broadcasting
\begin{frame}[fragile]
  Jaký je výstup následujícího kódu?
\begin{lstlisting}
import numpy as np
a = np.array([[1, 2, 3], [4, 5, 6]])
b = np.array([1, 2, 3])
print(a + b)
\end{lstlisting}
\begin{itemize}
  \item A)
  \begin{lstlisting}
    [[1 2 3]
     [4 5 6]
     [1 2 3]]
  \end{lstlisting}
  \item B)
  \begin{lstlisting}
    [[2 4 6]
     [4 5 6]]
  \end{lstlisting}
  \item C)
  \begin{lstlisting}
  [[2 4 6]
   [5 7 9]]
  \end{lstlisting}
\item D) Už nedokážu vymyslet další nesmyslné možnosti, ale správná odpověď je A
\end{itemize}
\end{frame}

\end{document}
