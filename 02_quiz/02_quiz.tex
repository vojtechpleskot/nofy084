\documentclass[9pt]{beamer}
%\usepackage[utf8]{inputenc}
%\usepackage[czech]{babel}
%\usepackage{slashed}
%\usepackage{axodraw}
%\usepackage{xcolor}
\usepackage{hyperref}
\hypersetup{unicode=true, colorlinks=true}
\usepackage{graphicx}
\usepackage{xspace}
%\usepackage{caption}
%\usepackage{subcaption}
%\usetheme{Warsaw}
\usepackage{variables}

\usepackage{listings}
\usepackage{xcolor}

\lstset{
    language=Python,
    basicstyle=\ttfamily\small,
    columns=flexible,         % Improves character spacing
    showspaces=false,          % Removes the u-shaped space symbols
    showstringspaces=false,    % Removes symbols inside 'strings'
    showtabs=false,            % Removes symbols for tabs
    breaklines=true,           % Wraps long lines
    keywordstyle=\color{blue},
    commentstyle=\color{gray},
    stringstyle=\color{orange}
}

\newenvironment{narrow}[2]{%
  \begin{list}{}{%
    \setlength{\topsep}{0pt}%
    \setlength{\leftmargin}{#1}%
    \setlength{\rightmargin}{#2}%
    \setlength{\listparindent}{\parindent}%
    \setlength{\itemindent}{\parindent}%
    \setlength{\parsep}{\parskip}%
  }%
  \item[]
}{\end{list}}

\title[]{Quiz}
\author[Vojtech Pleskot]{Vojtech Pleskot\inst{1}} %use \and to separate two authors
\institute[]{CU Prague\inst{1}}
\date{}

%\setbeamertemplate{headline}
%{
%  \begin{beamercolorbox}[wd=\paperwidth,ht=3.5ex,leftskip=.3cm,rightskip=.3cm]{section in head/foot}
%    \vskip2pt\insertsection \hskip20pt \insertsubsection\vskip2pt
%  \end{beamercolorbox}
%}

\setbeamertemplate{footline}
{
%  \hbox{\begin{beamercolorbox}[wd=.5\paperwidth,ht=2.5ex,dp=1.125ex,leftskip=.3cm plus1fill,rightskip=.3cm]{author in head/foot}
%    \usebeamerfont{author in head/foot}\insertauthor
%  \end{beamercolorbox}
  \begin{beamercolorbox}[wd=\paperwidth,ht=2.0ex,dp=3.0ex,leftskip=.3cm,rightskip=.3cm plus1fil]{title in head/foot}
    \usebeamerfont{title in head/foot} \hfill \insertframenumber/\inserttotalframenumber
  \end{beamercolorbox}%}
  \vskip0pt
}

\begin{document}
\begin{frame}
  \titlepage
\end{frame}

% This is a simple quiz with one question per frame. The correct answer is revealed on the next frame.
% For each question, there are four to six possible answers.
% The questions are numbered from A to F.
% There might be several correct answers to each question, but at least one of the answers is correct.
% The test checks the university students' knowledge of the Python programming language and its libraries, as well as their ability to use it for data analysis and visualization.
% In particular, work with strings, lists, dictionaries, sets, tuples, functions, classes, modules, packages, and libraries such as NumPy, Matplotlib, SciPy, and uncertainties is tested.


% Function definition question
\begin{frame}[fragile]
  What is the value of the variable \texttt{a} after executing the following code?
\begin{lstlisting}
def func(x):
    return x * 2
a = func(3)
\end{lstlisting}
\begin{itemize}
\item A) 5
\item B) 6
\item C) 9
\item D) 1.5
\end{itemize}
\end{frame}

% List indexing question
\begin{frame}[fragile]
  What is the output of the following code?
\begin{lstlisting}
my_list = [1, 2, 3]
print(my_list[2])
\end{lstlisting}
\begin{itemize}
\item A) [3]
\item B) [1, 2]
\item C) 3
\item D) IndexError: list index out of range
\end{itemize}
\end{frame}

% String formatting question
\begin{frame}[fragile]
  Let's have a variable \texttt{x} whose value is 10. Which string is equal to "The value of x is 10"?
\begin{itemize}
\item A) "The value of x is " + str(x)
\item B) "The value of x is 10"
\item C) "The value of x is " + x
\item D) f"The value of x is \{x\}"
\item E) "The value of x is \{\}".format(x)
\end{itemize}
\end{frame}

% Dictionary question
\begin{frame}[fragile]
  What is the output of the following code?
\begin{lstlisting}
my_dict = {'a': 1, 'b': 2, 'c': 3}
print(my_dict['b'])
\end{lstlisting}
\begin{itemize}
\item A) 'b'
\item B) 2
\item C) \{'b': 2\}
\item D) KeyError: 'b'
\end{itemize}
\end{frame}

% Variable scope question
\begin{frame}[fragile]
  What is the value of \texttt{x} after executing the following code?
\begin{lstlisting}
x = 10
def func():
    x = 5
    return x
a = func()
\end{lstlisting}
\begin{itemize}
\item A) None
\item B) 5
\item C) nan
\item D) 10
\end{itemize}
\end{frame}

% Set operations question
\begin{frame}[fragile]
  What is the output of the following code?
\begin{lstlisting}
my_set = {1, 2, 3}
my_set.add(3)
print(my_set)
\end{lstlisting}
\begin{itemize}
\item A) \{1, 2, 3, 3\}
\item B) AttributeError: 'set' object has no attribute 'add'
\item C) \{1, 2, 3\}
\item D) 'Jak vám dupou králíci?'
\end{itemize}
\end{frame}

% List slicing question
\begin{frame}[fragile]
  What is the output of the following code?
\begin{lstlisting}
my_list = [1, 2, 3, 4, 5]
print(my_list[1:4])
\end{lstlisting}
\begin{itemize}
\item A) [1, 2, 3]
\item B) [2, 3, 4]
\item C) [3, 4, 5]
\item D) [1, 2, 3, 4]
\item E) [2, 3, 4, 5]
\end{itemize}
\end{frame}

% Dictionary comprehension question
\begin{frame}[fragile]
  What is the output of the following code?
\begin{lstlisting}
my_dict = {x: x**2 for x in range(5)}
print(my_dict)
\end{lstlisting}
\begin{itemize}
\item A) \{1: 1, 2: 4, 3: 9, 4: 16\}
\item B) \{1: 1, 2: 4, 3: 9, 4: 16, 5: 25\}
\item C) \{0: 0, 1: 1, 2: 4, 3: 9, 4: 16\}
\item D) \{0: 0, 1: 1, 2: 4, 3: 9, 4: 16, 5: 25\}
\item E) \{(0, 1, 2, 3, 4): (0, 1, 4, 9, 16)\}
\end{itemize}
\end{frame}

% Tuple unpacking question
\begin{frame}[fragile]
  What is the output of the following code?
\begin{lstlisting}
def func():
    return 1, 2
a, b = func()
print(a, b)
\end{lstlisting}
\begin{itemize}
\item A) (1, 2)
\item B) 1 2
\item C) 1, 2
\item D) ValueError: too many values to unpack
\end{itemize}
\end{frame}

% Text file reading question
\begin{frame}[fragile]
  How to get the content of the file 'file.txt' read to the variable \texttt{data}?
\begin{itemize}
\item A) Sezame, otevri se!
\item B) data = open('file.txt', 'r')
\item C) with open('file.txt', 'r') as f: data = f.read()
\item D) f = 'file.txt'; data = f.read()
\item E) f = 'file.txt'; for line in f: data += line
\end{itemize}
\end{frame}

% Module import question
\begin{frame}[fragile]
  How to calculate \texttt{sin(1)} using the \texttt{math} module?
\begin{itemize}
\item A) import math; sin(1)
\item B) from math import sin; sin(1)
\item C) import math as m; math.sin(1)
\item D) import math; math.sin(1)
\item E) chodit do kurzu NOFY084, bedlivě poslouchat a udělat všechny úkoly
\end{itemize}
\end{frame}

% NumPy array addition question
\begin{frame}[fragile]
  What is the output of the following code?
\begin{lstlisting}
import numpy as np
a = np.array([1, 2, 3])
b = np.array([4, 5, 6])
print(a + b)
\end{lstlisting}
\begin{itemize}
  \item A) [1 2 3] + [4 5 6]
  \item B) [1 2 3 4 5 6]
  \item C) [4 5 6 1 2 3]
  \item D) [5 7 9]
\end{itemize}
\end{frame}

% NumPy array multiplication question
\begin{frame}[fragile]
  What is the output of the following code?
\begin{lstlisting}
  import numpy as np
  a = np.array([1, 2, 3])
  b = np.array([1, 2, 3])
  print(a * b)
\end{lstlisting}
\begin{itemize}
\item A)
\begin{lstlisting}
[[1 2 3]
 [1 2 3]
 [1 2 3]]
\end{lstlisting}
\item B) 14
\item C) [1 4 9]
\item D) [1 2 3] * [4 5 6]
\end{itemize}
\end{frame}

% Broadcasting question
\begin{frame}[fragile]
  What is the output of the following code?
\begin{lstlisting}
import numpy as np
a = np.array([[1, 2, 3], [4, 5, 6]])
b = np.array([1, 2, 3])
print(a + b)
\end{lstlisting}
\begin{itemize}
  \item A)
  \begin{lstlisting}
    [[1 2 3]
     [4 5 6]
     [1 2 3]]
  \end{lstlisting}
  \item B)
  \begin{lstlisting}
    [[2 4 6]
     [4 5 6]]
  \end{lstlisting}
  \item C)
  \begin{lstlisting}
  [[2 4 6]
   [5 7 9]]
  \end{lstlisting}
\item D) Už nedokážu vymyslet další nesmyslné možnosti, ale správná odpověď je A
\end{itemize}
\end{frame}

\end{document}




















%ukazky, jak se co dela
%
%fill all the frame width with three pictures:
%  \begin{narrow}{-1.3cm}{-1.1cm}
%    \begin{tabular}[h]{ccc}
%      \includegraphics[width=4.5cm]{}&
%      \hspace{-0.75cm}    \includegraphics[width=4.5cm]{}&
%      \hspace{-0.75cm}    \includegraphics[width=4.5cm]{}\\
%    \end{tabular}
%  \end{narrow}
%
%two pictures side by side:
%  \begin{figure}[h]
%    \centering
%    \includegraphics[width=5.5cm]{}
%    \includegraphics[width=5.5cm]{}
%    \caption{popisek obrazku}
%  \end{figure}
%
%two pictures one under the other
%  \begin{narrow}{-1.3cm}{-1.1cm}
%    \begin{tabular}[h]{c}
%      \includegraphics[width=9.5cm]{} \vspace{-0.15cm} \\
%      \includegraphics[width=9.5cm]{}
%    \end{tabular}
%  \end{narrow}
%
%several pictures, each with its own caption
%  \begin{figure}
%    \centering
%    \begin{subfigure}[b]{0.49\textwidth}
%      \centering
%      \caption{}
%      \includegraphics[width=5.5cm]{}
%    \end{subfigure}
%    \begin{subfigure}[b]{0.49\textwidth}
%      \centering
%      \caption{}
%      \includegraphics[width=5.5cm]{}
%    \end{subfigure}
%  \end{figure}
%
%tabulka:
%    \begin{table}[h]
%      \centering
%      \begin{tabular}{|c|c|c||c|c|c|}
%        \hline
%        \multicolumn{3}{|c||}{leptony} &
%        \multicolumn{3}{|c|}{kvarky} \\
%        \hline
%        e & $\mu$ & $\tau$ & u & c & t\\
%        \hline
%        $\bar{\nu}_e$ & $\bar{\nu}_\mu$ & $\bar{\nu}_\tau$ & d & s & b\\
%        \hline
%      \end{tabular}
%    \end{table}
%
%tucny text:
%\textbf{muj tucny text}
%
%rozdeleni framu na sloupce:
%  \begin{columns}[t] % contents are top vertically aligned
%    \column{0.5\textwidth}
%    sem uz prijde, co chci
%    \column{0.5\textwidth}
%    sem taky prijde, co chci
%  \end{columns}
%
%showing items one by one:
%\begin{itemize}
%\item neco
%  \pause
%\item druha polozka
%  \pause
%\item treti polozka
%\end{itemize}
%
%hodne obrazku na jednom slidu:
%  \begin{narrow}{-1.3cm}{-1.1cm}
%  \begin{tabular}[h]{ccc}
%    \includegraphics[width=4.1cm]{}&
%\hspace{-0.3cm}    \includegraphics[width=4.1cm]{}&
%\hspace{-0.3cm}    \includegraphics[width=4.1cm]{}\\
%    \includegraphics[width=4.1cm]{}&
%\hspace{-0.3cm}    \includegraphics[width=4.1cm]{}&
%\hspace{-0.3cm}    \includegraphics[width=4.1cm]{}\\
%  \end{tabular}
%  \end{narrow}
%
%five pictures and place for the sixth replaced by itemize
%
%  \vspace{-0.4cm}
%  \begin{narrow}{-1.3cm}{-1.1cm}
%    \begin{tabular}[h]{p{4.2cm}p{4.2cm}p{4.2cm}}
%      \includegraphics[width=4.5cm]{}&
%      \hspace{-0.45cm}    \includegraphics[width=4.5cm]{}&
%      \hspace{-0.75cm}    \includegraphics[width=4.5cm]{}\\
%      \includegraphics[width=4.5cm]{}&
%      \hspace{-0.45cm}    \includegraphics[width=4.5cm]{}&
%      \vspace{-4.75cm}
%      \begin{narrow}{-0.8cm}{0.6cm}
%        \begin{itemize}
%        \item 
%        \end{itemize}
%      \end{narrow}\\  
%    \end{tabular}
%  \end{narrow}  
%
%merge rows in tabular: multirow usage example
%  \begin{table}[ph]
%    \centering
%    \begin{tabular}{|c|c|c|}
%      \hline
%      1.1&1.1&\multirow{3}{*}{merged}\\
%      \cline{1-2}
%      2.5 & 1 & \\
%      \cline{1-2}
%      2.5 & 1 & \\
%      \cline{1-3}
%      2.5 & 1 & 1\\
%      \cline{1-3}
%    \end{tabular}
%    \caption{}
%    \label{}
%  \end{table}
%
%
